\section[short]{Introduction}

Network simulation has been a popular tool for network research for a long time. It allows testing new protocols and algorithms in a controlled environment. The NS3 network simulator is a popular open source discrete event network simulator. It is written in C++ and provides a Python interface. It is used in many research projects and is under active development. The NS3 network simulator provides a large number of models for different network layers and protocols. The NS3 network simulator provides a large number of models for different network layers and protocols. The NS3 network simulator provides a large number of models for different network layers and protocols. The NS3 network simulator provides a large number of models for different network layers and protocols. The NS3 network simulator provides a large number of models for different network layers and protocols. The NS3 network simulator provides a large number of models for different network layers and protocols. The NS3 network simulator provides a large number of models for different network layers and protocols. The NS3 network simulator provides a large number of models for different network layers and protocols. The NS3 network simulator provides a large number of models for different network layers and protocols. The NS3 network simulator provides a large number of models for different network layers and protocols.
In order to model how signal propagation effects wireless communication, different propagation models have been 
developed. The NS3 network simulator provides many models for different network layers and protocols. The NS3 network simulator provides many models for different 
use cases and environments. In this work, some of the ones most applicable for 802.11n have been selected and compared at different 
distances with respect to throughput and received signal power. Distance is one of the most common factors for Rx power loss in wireless communications, since 
the spread of the electromagnetic waves across an increasing area with increasing distance means a lower power density at any given point of the 
area. Additionally, effects like reflection, refraction, absorption and interference can influence the wireless transmission performance.
Depending on the propagation model, these effects are simulated in some capacity. However, there is no consensus in the scientific community on 
which model is most suitable for which type of simulation. Therefore, the author compared a selection of them in a simple setting to have an idea 
on how each of them stacks up against each other and reality.
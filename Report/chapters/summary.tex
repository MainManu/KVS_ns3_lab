\section{Summary}
\label{ch:summary}

In this paper, the performance of the IEEE 802.11n standard over 
increasing distance using different propagation models was investigated using the NS3 (version 3.39) network simulator.
It was found that for a simple ad hoc network between 2 nodes without any interference, 
3 seconds suffice to reach a steady state. 
Additionally, the results show that the propagation model has a significant impact on the data rate and the received power.
Furthermore, it was found that special care has to be taken when parametrizing the models if taking 
advantage of the variety of offered propagation models is desired.
Future work should look at the parameters of the different propagation models and how they affect 
range and data rate in more detail. The influence of different topologies and interference should also be investigated.
However, for some models data rates above what the application produces was observed at the appp layer  between 50 and 100 meters. This should be 
investigated in future work.
Shortly before the submission of this paper, it was noticed the ns3 script did not specify a random seed. 
This means that the results depend on the default random seed of the ns3 simulator and all simulations 
which are run with identical parameters will produce the same results. This is a common pitfall in ns3 simulations \cite{10.1145/1096166.1096174},
and should be avoided in future work. 
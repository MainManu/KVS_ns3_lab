\begin{abstract}
Network simulation plays a crucial role in evaluating and testing protocols and algorithms
within a controlled environment. This paper focuses on the NS3 network simulator, 
a widely-used open-source tool for discrete event network simulations. Specifically, 
the study compares various propagation loss models within NS3 for the IEEE 802.11n standard, examining their impact on wireless communication in a simple point-to-point link scenario.
The selected models include FriisPropagationLossModel, FixedRssLossModel, 
ThreeLogDistancePropagationLossModel, TwoRayGroundPropagationLossModel, and NakagamiPropagationLossModel. To conduct the comparisons, a custom NS3 script was developed, incorporating a Minstrel HT manager,
Rx power and datarate monitoring, and UdpClientServerHelper for traffic generation. Simulations were executed remotely due to computational resource requirements, and a Python wrapper 
facilitated remote experiment control and data analysis.
The results indicate that the choice of the model significantly 
influences both data rate and received signal power. The study establishes a stable 
state after 3 seconds of simulation, providing a basis for meaningful data rate comparisons. 
Unexpected spikes in data rate for certain models between 50 and 100 meters warrant further 
investigation. The Rx power results highlight similarities between specific models, urging a 
deeper examination of their configurations and parameter values in future work.
This research contributes valuable insights into the performance of the IEEE 802.11n 
standard over varying distances, emphasizing the importance of propagation model selection. 
It underscores the need for careful parameterization when utilizing diverse propagation models. 
Future work should delve into the detailed impact of model parameters on range and data rate, 
exploring different network topologies and interference scenarios.

\end{abstract}


\begin{IEEEkeywords}
NS3, propagation loss models, 802.11n, network simulation, wireless communication
\end{IEEEkeywords}
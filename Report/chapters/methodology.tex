\section[short]{Methodology}

First of all, suitable propagation models had to be selected.
NS3 supports a multitude of propagation models which can be found on \cite[the ns3 website]{ns3prpmod}.
However for this work only the following models were deemed suitable:
\begin{itemize}
    \item FriisPropagationLossModel
    \item FixedRssLossModel
    \item ThreeLogDistancePropagationLossModel
    \item TwoRayGroundPropagationLossModel
    \item NakagamiPropagationLossModel
\end{itemize}
In order to compare them, a ns3 script was set up. It was very similar to the script wifi-spectrum-per-example.cc 
with some notable changes: Instead of a constant rate Wi-Fi manager (which would make the analysis of the data rate 
pointless), a Minstrel HT manager was used. 
Furthermore, callbacks for monitoring the Rx power were added, including some tools for value conversions and
averaging. Some command line arguments like simulationTime, disctance, propagationModel, etc. were added as well.
Since these simulations required substantial computational resources, they were run on a remote server. 
For easy experiment control and file management, a python wrapper script was written. It allowed to run an experiment 
and download all the output files automatically. This wrapper was then imported into a jupyter notebook for 
testing and data analysis. The data analysis was then performed in the notebook using pandas and matplotlib.
Since the simulation contains non-deterministic processes and a steady state siimulation is desired,
the time until a steady state is reached first had to be determined. This was done by running a sample simiulation 
with one propagation model, distance and a long simulation time. Then the throughput and rx power were plotted over
time. The time until the throughput and rx power reached a steady state was determined by visual inspection.
After that, the simulation was run 5 times and the confidence intervals were calculated to verify the steady 
state assumption.



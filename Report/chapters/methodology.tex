\section[short]{Methodology}

First of all, suitable propagation models had to be selected.
NS3 supports a multitude of propagation models which can be found on \cite[the ns3 website]{ns3prpmod}.
However for this work only the following models were deemed suitable:
\begin{itemize}
    \item FriisPropagationLossModel
    \item FixedRssLossModel
    \item ThreeLogDistancePropagationLossModel
    \item TwoRayGroundPropagationLossModel
    \item NakagamiPropagationLossModel
\end{itemize}
In order to compare them, a ns3 script was set up. It was very similar to the script wifi-spectrum-per-example.cc 
with some notable changes: Instead of a constant rate Wi-Fi manager (which would make the analysis of the data rate 
pointless), a Minstrel HT manager was used. 
Furthermore, callbacks for monitoring the Rx power were added, including some tools for value conversions and
averaging. Some command line arguments like simulationTime, disctance, propagationModel, etc. were added as well.
Additionally, callbacks were added to the script to save the Rx power and throughput to a file.
Both of these saved the time and value of the respective variable in a std::map. At the end of the simulation, 
both maps were exported to csv files. This would make importing them into pandas trivial.
Traffic was generated at the application layer using a UdpClientServerHelper set to a constant data rate of 75 Mbps.
Since these simulations required substantial computational resources, they were run on a remote server. 
For easy experiment control and file management, a python wrapper script was written. It allowed to run an experiment 
and download all the output files automatically. This wrapper was then imported into a jupyter notebook for 
testing and data analysis. The data analysis was then performed in the notebook using pandas and matplotlib.
The source code for the ns3 script, the python wrapper and the jupyter notebooks can be found on \cite[the author's github]{github}.
If you wish to reproduce the results, please follow the instructions in the README.md file on how to 
set up ns3.39, symlink the repositories scratch folder into ns3, set up the python environment, set up the 
environment variables and run the jupyter notebooks.
%TODO: add expected results with sources
The expectations for the results of the different simulaitons were quite different.
The FixedRssLossModel isolates the propagation delay as the only effect of the distance on the transmission. This means 
that the data rate should be constant until the distance is so large that the propagation delay causes issues with the 
WiFi protocol timings. This should result in a sudden drop in the datarate as soon as this 
critical distance is reached. The Rx power per definition should stay constant.
For the other models, the data rate should stay constant until the Rx power drops far enough that the current Htmcs value is 
no longer sufficient to maintain the data rate. This should result in a sudden drop in the data rate down to the datarate assiciated with the 
next Htmcs value.